\documentclass[a4paper,12pt]{article}
\usepackage[margin=1in]{geometry} % Adjust margins if needed
\usepackage{ulem} % Importar el paquete ulem
\usepackage{graphicx}
\usepackage{float}
\usepackage{parskip}
\usepackage{caption}
\usepackage{subcaption}
\usepackage{caption}
\usepackage{siunitx}
\usepackage{graphicx}
\usepackage{circuitikz,siunitx}
\usepackage{tikz}

\captionsetup[figure]{name=Fig.} % Change the label for figures

\title{TP4 - Teoría de Circuitos 1\\ Acoplamiento Magnético y Cuadripolos}

\author{Autores: \\Pla, Juan Ignacio (63486)\\Torino, Joaquín (63140)\\Caviglia, Facundo (63178)\\Belsito, Ramiro (62641)}
\date{Actualizado: \today}

\begin{document}
\maketitle

\section{Introducción}
\hspace{1cm}

\hspace{1cm}

\hspace{1cm}
\hspace{1cm}

\section{Materiales utilizados}

\begin{itemize}
\item
\item 
\end{itemize}

\section{Desarrollo}

\hspace{1cm} El siguiente fue el circuito utilizado para las 4 partes de la experiencia.

\end{document}